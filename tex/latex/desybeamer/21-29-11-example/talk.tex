\ifdefined\handout
	\documentclass[lualatex,handout,aspectratio=1610]{beamer}
\else
	\documentclass[lualatex,beamer,aspectratio=1610,dvipsnames]{beamer}
\fi


\usetheme{DESY}


\newcommand{\TITLE}{Title}
\newcommand{\SUBTITLE}{Subtitle}
\newcommand{\AUTHOR}{
	Author 1, Author 2, Author 3,\break
	Author 4, \textbf{Presenter}, Author 6}
\newcommand{\WORKSHOP}{Workshop}
\newcommand{\COLLABORATION}{Collaboration}
\newcommand{\DATE}{\today}

\usepackage[sectionslide]{desybeamer}

\toggletrue{colbracket}

% pdfpc plugin
% https://github.com/cebe/pdfpc-latex-notes
\usepackage[duration=10:00,starttime=00:00,lastminutes=1]{pdfpcnotes}


\begin{document}

\maketitle


\begin{frame}{Vision}
	\begin{center}
		\large Save CPU time by using a \textbf{Neural Network} to simulate the \textbf{HGCAL}.
	\end{center}
	\vfill
	% \begin{center}
	% \includegraphics[keepaspectratio,width=\textwidth,height=.95\textheight]{ganslide-rot.pdf}
	% \end{center}
	\pause
	\begin{center}
		\small Use Graph Neural Networks (“GNNs”) to deal with sparsity and irregular geometry
	\end{center}

\end{frame}



\begin{frame}{Design Decisions}
	\begin{itemize}[<+->]
		\item Stay as close as possible to the detector geometry \imp Node ≡ Cell
		\item Q1: When is the neighborhood between the cells constructed?\\
		      \begin{itemize}
			      \item “Post simulation” step: construct the neighborhood for each event (in CMSSW)
			            \begin{itemize}
				            \item simulation always matches geometry
			            \end{itemize}
			            \pause\vs
			      \item Preprocessing step: read neighborhood from lookup table
			            \begin{itemize}
				            \item faster $⇐$
			            \end{itemize}
		      \end{itemize}
		\item Q2: How do we continue after the simulation?% In CMSSW \vs In a python environment?
	\end{itemize}
\end{frame}


\begin{frame}{Simulation and Machine Learning Frameworks}
	\begin{columns}
		\begin{column}{.5\textwidth}
			\textbf{Integration}: Simulate in CMSSW, use the integrated ML software stack.
		\end{column}
		\begin{column}{.5\textwidth}
			\textbf{Decoupling}: Simulate in CMSSW, continue with standard ML software stack.
		\end{column}
	\end{columns}
	\pause
	\begin{columns}
		\begin{column}{.5\textwidth}
			\begin{itemize}
				\item Directly integrated in the production process\pause
			\end{itemize}
		\end{column}
		\begin{column}{.5\textwidth}
			\begin{itemize}
				\item Lean development environment\pause
				\item Faster development cycle\pause
				\item Easier onboarding\pause
				\item Access to HPC clusters w/o CMS software available\pause
				\item Cutting edge software versions \pause
			\end{itemize}
			\imp Preferred solution for this case
		\end{column}
	\end{columns}

	\vfill
	\pause
	\begin{center}
		The integrated ML tool chain in CMSSW is also widely used in CMS!
	\end{center}
\end{frame}


% \section{Data Loader}
\begin{frame}{Loading the dataset}
	\begin{columns}
		\begin{column}{.65\textwidth}
			\begin{enumerate}
				\item Read simulated hits from ROOT file\tikzmark{topdl}
				      \begin{itemize}
					      \item Most tools cannot handle data of variable dimension
					      \item \texttt{uproot} → \texttt{awkward} arrays
				      \end{itemize}
				      \pause
				\item Convert simulated hits to graphs
				      \begin{itemize}
					      \item Select the active cells from the extracted geometry
					      \item Extract the cell properties, construct the neighborhood information\\
					            CPU intensive \imp parallel processing
				      \end{itemize}
				      \pause
				\item Batch the graphs (\texttt{torch\_geometric}~\cite{torch_geometric.batch})\tikzmark{bottomdl}
				      \pause
				\item Move to GPU
			\end{enumerate}
		\end{column}
		\begin{column}{.33\textwidth}
			\tikzmark{rightdl}
		\end{column}
	\end{columns}
	\pause
	\InsertRightBrace{topdl}{bottomdl}{rightdl}{Custom dataloader\\ based on\\ \small\texttt{torch.multiprocessing}~\cite{torch.multiprocessing}}
	\pnote{
		1. number of events x number of hits\\
		⇒ variable number of dimensions\\
		⇒ tools fail\\
		scihep software stack\\
		2. 3.7M Cells/ Endcap to much \\
		⇒ prune cells w/o deposits\\
		3. gpu utilisation ⇒ batching ⇒ torch_geometric \\
		4.dont reprocess for each epoch
	}
\end{frame}


\begin{frame}{Loading the dataset}
	\label{dataloder}
	\begin{columns}
		\begin{column}{.65\textwidth}
			\begin{enumerate}
				\item Read Simulated hits from ROOT file\tikzmark{topdl}
				      \begin{itemize}
					      \item Most tools cannot handle data of variable dimension
					      \item \texttt{uproot} → \texttt{akward} arrays
				      \end{itemize}
				\item Convert simulated hits to graphs
				      \begin{itemize}
					      \item Select the active cells from the extracted geometry
					      \item Extract the cell properties, construct the neighborhood information\\
					            CPU intensive \imp parallel processing
				      \end{itemize}
				\item Batch the graphs (\texttt{torch\_geometric}~\cite{torch_geometric.batch})\tikzmark{bottomdl}
				\item (De)Serialize files (from) to disc (\texttt{torch.save})
				\item Move to GPU
			\end{enumerate}
		\end{column}
		\begin{column}{.33\textwidth}
			\tikzmark{rightdl}
		\end{column}
	\end{columns}
	\InsertRightBrace{topdl}{bottomdl}{rightdl}{Custom dataloader\\ based on\\ \small\texttt{torch.multiprocessing}~\cite{torch.multiprocessing}}
\end{frame}


\thankyou

\begin{frame}[allowframebreaks]{Bibliography}
	\printbibliography
\end{frame}

\mode<beamer>

\appendix

\section{Backup}


\begin{frame}[t,fragile]{Control data flow with processes pools and queues}
	\begin{python}
		pseq = Sequence(
		ProcessStep(read_chunk, 2),
		PoolStep(simhits_to_graph,nworkers=20),
		RepackStep(batch_size),
		ProcessStep(torch_geometric.data.Batch().from_data_list),
		Queue(prefetch_batches)
		)
		...
		pseq.queue_iterable(epoch_chunks)
		...
		for batch in pseq(filelist):
		prediction = model(batch)
		...
	\end{python}
	\begin{itemize}
		\item fast
		\item limits memory usage
		\item option to save to disk
	\end{itemize}
\end{frame}

\end{document}
